\documentclass[10pt]{article}

\usepackage{graphicx}
\usepackage[top=0.25in, left=1.0in, right=1.0in]{geometry}
\setlength{\topmargin}{-1.0in}
\usepackage{url}
\usepackage{enumitem}
\usepackage{tabularx}
\usepackage{setspace}
\usepackage{tikz}
\usetikzlibrary{er,positioning}  
\usetikzlibrary{decorations.pathreplacing}
\usetikzlibrary{arrows, decorations.markings}
\usetikzlibrary{shapes.geometric}
\usepackage{amsmath}
\usepackage{hyperref}

\usepackage[spanish]{babel}
\usepackage[utf8]{inputenc}

\usepackage{datetime}
\newdate{date}{16}{12}{2024}
\date{\displaydate{date}}

\usepackage{listingsutf8}
\usepackage{xcolor,listings}
\usepackage{textcomp}
\usepackage{color}

\usepackage{dirtree}

\definecolor{codegreen}{rgb}{0,0.6,0}
\definecolor{codegray}{rgb}{0.5,0.5,0.5}
\definecolor{codepurple}{HTML}{C42043}
\definecolor{backcolour}{HTML}{F2F2F2}
\definecolor{red}{HTML}{FF0000}
\definecolor{bookColor}{cmyk}{0,0,0,0.90}  
\color{bookColor}

\lstset{upquote=true}

\lstdefinestyle{mystyle}{
	% backgroundcolor=\color{backcolour},   
	commentstyle=\color{codegreen},
	keywordstyle=\color{codepurple},
	numberstyle=\numberstyle,
	stringstyle=\color{codepurple},
	basicstyle=\footnotesize\ttfamily,
	breakatwhitespace=false,
	breaklines=true,
	captionpos=b,
	keepspaces=true,
	% numbers=left,
	numbersep=10pt,
	showspaces=false,
	showstringspaces=false,
	showtabs=false,
	otherkeywords = {OUT, SIGNAL, DECLARE, BEFORE, before, AFTER, CALL, WHILE, PROCEDURE, if, if},
}
\lstset{style=mystyle}

\newcommand\numberstyle[1]{%
	\footnotesize
	\color{codegray}%
	\ttfamily
	\ifnum#1<10 0\fi#1 |%
}

\newcommand{\linetomark}{\rule{0.5cm}{0.4pt} }

\begin{document}
	
	\lstset{
		literate=%
		{á}{{\'a}}1
		{í}{{\'i}}1
		{é}{{\'e}}1
		{ý}{{\'y}}1
		{ú}{{\'u}}1
		{ó}{{\'o}}1
		{ě}{{\v{e}}}1
		{š}{{\v{s}}}1
		{č}{{\v{c}}}1
		{ř}{{\v{r}}}1
		{ž}{{\v{z}}}1
		{ď}{{\v{d}}}1
		{ť}{{\v{t}}}1
		{ň}{{\v{n}}}1                
		{ů}{{\r{u}}}1
		{Á}{{\'A}}1
		{Í}{{\'I}}1
		{É}{{\'E}}1
		{Ý}{{\'Y}}1
		{Ú}{{\'U}}1
		{Ó}{{\'O}}1
		{Ě}{{\v{E}}}1
		{Š}{{\v{S}}}1
		{Č}{{\v{C}}}1
		{Ř}{{\v{R}}}1
		{Ž}{{\v{Z}}}1
		{Ď}{{\v{D}}}1
		{Ť}{{\v{T}}}1
		{Ň}{{\v{N}}}1                
		{Ů}{{\r{U}}}1    
	}
	
	\begin{centering}
				
		\huge Orientación de Proyecto \\[2mm]
		
		\small Bases de Datos\\
		Licenciatura en Ciencia de Datos\\
		Curso: 2025-2026
		
	\end{centering}
	
	\vspace{2mm}
	
	Este documento proporciona una orientación para el proyecto final de la asignatura. \textbf{La realización de este proyecto constituye un ejercicio individual}.
	
	\section*{Escenario}
	
	Usted ha sido contratado como analista de datos en la empresa GreenSolutions, creadores de la plataforma GreenScape. Debido a despidos masivos, todo el personal relacionado con el desarrollo de GreenScape ya no se encuentra disponible. Sin embargo, con vistas a mejorar la plataforma en el futuro, la empresa le ha encargado obtener métricas acerca de su utilización por los usuarios y realizar una propuesta de mejora. 
	
	De forma individual, y sin contar con ayuda de nadie en la empresa que conozca la base de datos de GreenScape, usted deberá completar las tareas asignadas por la empresa.
	
	\section*{Materiales provistos}
	
	Para el desarrollo del proyecto se le provee de los siguientes recursos:
	
	\begin{itemize}
		\item \textbf{Imagen Docker}: Una imagen de Docker con la base de datos que la empresa cuenta. Esta base de datos se encuentra vacía debido a problemas de seguridad actuales.
		
		\item \textbf{Documento de orientación}: El presente documento con los requerimientos y las tareas a desarrollar en el plazo máximo definido.
		
		\item \textbf{Conjuntos de datos}: A mitad del período académico se brindará un fichero con conjuntos de datos que podrán utilizarse para poblar la base de datos, una vez que se complete el proceso de limpieza y anonimización de los datos sensibles de la empresa.
	\end{itemize}
	
	\section*{Tareas}
	
	\begin{enumerate}
		\item A partir de la base de datos proporcionada, elabore un \textbf{modelo entidad-relación extendido (MERX)} que le permita comprender conceptualmente el escenario planteado. 
		Puede utilizar cualquier herramienta para su construcción; sin embargo, se recomienda emplear 
		\href{https://app.diagrams.net/?src=about}{diagrams.net} o 
		\href{https://excalidraw.com/}{Excalidraw}.
		
		\item Escribir código MySQL que responda a los siguientes ejercicios:
		
		\begin{enumerate}
            \item \textbf{Listar todos los productos disponibles}: Muestra todos los productos registrados, incluyendo sus detalles.
        
            \item \textbf{Contar las reacciones por publicación}: Calcula el número total de reacciones recibidas por cada publicación, incluyendo información sobre la publicación y su autor.
        
            \item \textbf{Tipos de plantas más gustadas}: Lista los tres tipos de plantas que han recibido la mayor cantidad de reacciones positivas.
        
            \item \textbf{Usuarios activos en contribuciones y reacciones}: Determina la actividad de los usuarios mostrando, para cada uno, su información personal y la última fecha (dentro de los últimos seis meses) en que realizó una contribución o reaccionó a una publicación. Si el usuario no presenta actividad reciente, la fecha debe mostrarse como \texttt{NULL}.
        
            \item \textbf{Publicaciones más populares considerando las reacciones}: Identifica las publicaciones que presentan una mayor cantidad de reacciones positivas que negativas, incluyendo sus detalles y el número total de reacciones.
        
            \item \textbf{Contribuciones constantes}: Muestra todas las plantas que han recibido contribuciones en dos meses consecutivos.
        
            \item \textbf{Promedio de actividad}: Determina el promedio de actividad mensual de los diez usuarios más activos en agregar contenido multimedia a sus publicaciones durante el último año (promedio de videos/fotos agregados por mes).
        
            \item \textbf{Distribución de edades}: Analiza la distribución de los usuarios por rangos de edad de diez años (por ejemplo: 11–20, 21–30, 31–40, 41–50, ...). Devuelve la cantidad de usuarios y el por ciento de edad por categoría.
        
            \item \textbf{Productos sin incremento en ventas mensuales}: Identifica los productos que no han mostrado un incremento en sus ventas mes a mes durante el último año.
        
            \item \textbf{Tendencias de contribución según el clima}: Examina cómo varían las contribuciones de acuerdo con el tipo de clima, identificando la planta más popular en cada categoría climática.
        
            \item \textbf{Cambio de preferencia en categorías de plantas}: Identifica a los usuarios que han cambiado su categoría de planta más contribuida al comparar la actividad entre dos años consecutivos.
        
            \item \textbf{Compras contradictorias}: Destaca a los usuarios que han comprado más plantas no marcadas como “Me gusta” que plantas marcadas como tales.
        
            \item \textbf{Usuarios de solo texto}: Lista los usuarios que nunca han agregado contenido multimedia a sus publicaciones.
        
            \item \textbf{Vendedores mejor calificados}: Muestra los cinco vendedores con mejor calificación promedio, ordenados de forma descendente. De cada uno se debe incluir el total de productos vendidos, calificación promedio, nombre, correo electrónico y dirección particular.
        
             \item \textbf{Compradores de solo plantas}: Identifica a los usuarios cuyas compras siempre incluyen, al menos, una planta.
             
             \item \textbf{\emph{Trigger} de auditoría de precios}: Implemente un disparador que se active automáticamente cuando se modifique el precio de un producto en la tabla \texttt{Producto}. Este recurso debe:
             \begin{itemize}
             	\item Registrar en una tabla de auditoría (\texttt{Historial\_Precios}) la información del cambio: producto afectado, precio anterior, precio nuevo, fecha y hora del cambio.
             	\item Calcular el porcentaje de cambio en el precio.
             	\item La tabla de auditoría debe ser creada con la estructura apropiada para almacenar estos datos.
             \end{itemize}
             
             \item \textbf{Procedimiento almacenado para análisis de actividad de usuario}: Cree un procedimiento almacenado llamado \texttt{sp\_analisis\_usuario} que reciba como parámetros:
             \begin{itemize}
             	\item \texttt{p\_id\_usuario}: ID del usuario a analizar
             	\item \texttt{p\_fecha\_inicio}: Fecha inicial del período de análisis
             	\item \texttt{p\_fecha\_fin}: Fecha final del período de análisis
             \end{itemize}
             El procedimiento debe retornar un conjunto de resultados que incluya:
             \begin{itemize}
             	\item Total de publicaciones realizadas en el período
             	\item Total de reacciones dadas y recibidas
             	\item Total de comentarios realizados
             	\item Total de compras y monto gastado
             	\item Total de contribuciones realizadas
             	\item Planta más comprada y planta más contribuida
             \end{itemize}
             Este procedimiento debe ser invocable desde la aplicación Streamlit, permitiendo al usuario introducir los parámetros de forma interactiva y visualizar los resultados en un formato amigable.
             
             \item \textbf{Análisis de influencers y su impacto en ventas}: Identifique a los 5 usuarios ``\emph{influencers}'' (aquellos cuyas publicaciones generan más interacciones) y determine si existe correlación entre su actividad y las ventas de las plantas con las que interactúan. Para cada \emph{influencer}, calcule: 
             \begin{enumerate}
				\item Su puntaje de interacciones ponderado (Me gusta=$1$, Me encanta=$2$, Me asombra=$1.5$, comentarios=$2$); 
				\item Las plantas con las que más ha interactuado (publicaciones, contribuciones, compras); 
				\item El incremento porcentual en ventas de esas plantas en las $2$ semanas posteriores a sus publicaciones/contribuciones versus las $2$ semanas anteriores; 
				\item La tasa de conversión: porcentaje de usuarios que compraron plantas después de reaccionar a sus publicaciones.
             \end{enumerate}
             
             \item \textbf{Detección de patrones de comportamiento anómalo en vendedores}: Identifique vendedores con patrones de venta sospechosos mediante el análisis de: 
             \begin{enumerate}
             	\item Vendedores que venden el mismo producto a precios significativamente diferentes ($>30\%$ de variación) sin justificación temporal; 
             	\item Vendedores con productos que tienen calificaciones extremadamente polarizadas (muchos $5$ y muchos $1$, pero pocos valores intermedios); 
             	\item  Vendedores cuyos compradores nunca o raramente han comprado otros productos en la plataforma (posible manipulación); 
             	\item Productos vendidos por múltiples vendedores donde uno tiene un patrón de ventas muy diferente al resto. Para cada vendedor sospechoso, genere un ``índice de sospecha'' ponderado y liste las evidencias específicas detectadas.
             \end{enumerate}
             
         \end{enumerate}
        
		
		\item Modifique el diseño de la base de datos al permitir conversaciones en los comentarios, es decir, permitir que los usuarios puedan responder comentarios de otros usuarios creando hilos o conversaciones.
		
		\begin{itemize}
			\item Escriba el código para modificar la base de datos de acuerdo a su diseño y genere conversaciones de prueba de longitud no menor que 20.
			
			\item Escriba código para dado un comentario inicial obtener la conversación entera surgida a partir de dicho comentario.
		\end{itemize}
		
		\item Modele y resuelva el problema de conversaciones en comentarios utilizando otro modelo de datos visto en conferencias.
		
		\item Compare ambos diseños, relacional y no relacional, de acuerdo a las ventajas y desventajas que presentan para resolver este escenario.
		
		\item Implemente un sistema de documentación jerárquica para las plantas. Cada planta debe tener asociado un \textbf{documento principal} (Ficha Técnica), que contiene información esencial sobre el tipo de planta y sus cuidados necesarios. Además, cada documento principal puede tener asociados múltiples \textbf{documentos secundarios} que complementan la información, tales como:
		
		\begin{itemize}
			\item \textbf{Certificado Fitosanitario}: Información sobre el estado de salud y ausencia de plagas de la planta.
			\item \textbf{Guía de Riego Estacional}: Instrucciones específicas de riego según la estación del año.
			\item \textbf{Manual de Tratamiento de Plagas}: Procedimientos para prevenir y tratar plagas comunes.
			\item \textbf{Historial de Crecimiento}: Registro del desarrollo esperado de la planta en diferentes etapas.
			\item \textbf{Análisis de Suelo}: Especificaciones detalladas sobre el tipo de suelo y nutrientes requeridos.
		\end{itemize}
		
		Para esta tarea debe:
		
		\begin{itemize}
			\item Modificar el diseño de la base de datos para permitir la incorporación de documentos principales asociados a plantas y documentos secundarios vinculados a los principales.
			\item Implementar el código necesario para incorporar ambos tipos de documentos al sistema.
			\item Demostrar el funcionamiento mediante la inserción de al menos $5$ fichas técnicas con sus respectivos documentos secundarios (mínimo $3$ documentos secundarios por ficha).
			\item Escribir una consulta que, a partir de una planta, obtenga todos sus documentos asociados (principal y secundarios) de forma jerárquica.
		\end{itemize}
		
		\item Desarrolle una aplicación web interactiva utilizando \textbf{Streamlit} que proporcione una interfaz gráfica para interactuar con la base de datos GreenScape. La aplicación debe incluir las siguientes funcionalidades:
		
		\begin{itemize}
			\item \textbf{Selector de consultas}: Un desplegable que permita seleccionar cualquiera de las consultas SQL implementadas en las tareas anteriores, mostrando los resultados en una tabla interactiva.
			
			\item \textbf{Análisis de usuario con procedimiento almacenado}: Interfaz interactiva que permita al usuario introducir los parámetros del procedimiento almacenado \texttt{sp\_analisis\_usuario} (ID de usuario, fecha inicio, fecha fin) y visualizar los resultados del análisis de actividad en formato amigable (tablas, gráficos, métricas) según convenga.
			
			\item \textbf{Gestión de conversaciones}: Interfaz para crear nuevos comentarios y asociarlos a hilos de conversación existentes, permitiendo la navegación jerárquica de las conversaciones.
			
			\item \textbf{Explorador de documentos}: Funcionalidad para seleccionar una planta y visualizar todos sus documentos asociados (principales y secundarios) de forma organizada y jerárquica.
			
			\item \textbf{Gestor de precios de productos}: Interfaz que permita seleccionar un producto, visualizar su precio actual y modificarlo. Esta funcionalidad debe:
			\begin{itemize}
				\item Mostrar una lista de productos con sus precios actuales
				\item Permitir al usuario seleccionar un producto y cambiar su precio
				\item Después de actualizar el precio, consultar y mostrar el historial de cambios de precio registrado por el trigger de auditoría, demostrando que el trigger se activó correctamente
				\item Visualizar la tabla de auditoría con los cambios históricos de precios para el producto seleccionado
			\end{itemize}
		\end{itemize}
		
		La aplicación debe ser completamente funcional, con manejo de errores robusto y una interfaz intuitiva que facilite la interacción con los datos de GreenScape.
		
	\end{enumerate}
	
	\section*{Restricciones Técnicas}
	
	Para el desarrollo de la aplicación Streamlit y la interacción con la base de datos, se deben cumplir las siguientes restricciones técnicas:
	
	\begin{itemize}
		\item \textbf{Prohibido el uso de ORM}: No se permite utilizar bibliotecas ORM (como SQLAlchemy, Django ORM, Peewee, etc.) para el trabajo con la base de datos. El objetivo es que los estudiantes trabajen directamente con SQL y comprendan las operaciones de base de datos a bajo nivel.
		
		\item \textbf{Uso obligatorio de bibliotecas SQL nativas}: Debe utilizarse bibliotecas de Python que permitan definir la sintaxis SQL directamente, tales como:
		\begin{itemize}
			\item \texttt{mysql-connector-python}
			\item \texttt{pymysql}
			\item \texttt{psycopg2} (para PostgreSQL)
			\item \texttt{sqlite3} (módulo estándar de Python)
		\end{itemize}
		
		\item \textbf{Consultas SQL explícitas}: Todas las operaciones de base de datos deben realizarse mediante consultas SQL escritas explícitamente en el código, sin abstracciones de alto nivel. Esto incluye operaciones de SELECT, INSERT, UPDATE, DELETE, y llamadas a procedimientos almacenados.
		
	\end{itemize}
	
	\section*{Entrega y Defensa}
	
	La entrega y discusión del proyecto se realizará en la \textbf{semana 16} del curso académico, o en fechas posteriores si así se decide y comunica oportunamente.
	
	\section*{Formato de entrega}
	
	El estudiante deberá preparar un \textbf{repositorio} (público o privado con acceso compartido al profesor) que contenga:
	
	\begin{itemize}
		\item \textbf{Código SQL}: Todos los scripts necesarios para la creación, modificación y consulta de la base de datos, organizados por tarea en carpetas separadas.
		
		\item \textbf{Modelo MERX}: Documento en formato \texttt{.pdf} o imagen que muestre el modelo entidad-relación extendido de la base de datos original.
		
		\item \textbf{Código de consultas}: Archivo \texttt{.sql} o notebook con las soluciones a los ejercicios planteados, debidamente comentados (si aplica).
		
		\item \textbf{Documentación NoSQL}: Scripts, notebooks o archivos de configuración correspondientes a la implementación en el modelo de datos no relacional seleccionado (si aplica).
		
		\item \textbf{Informe comparativo}: Documento en formato \texttt{.pdf} que contenga la comparación entre los diseños relacional y no relacional, incluyendo ventajas, desventajas y conclusiones.
		
		\item \textbf{Aplicación Streamlit}: Código fuente completo de la aplicación web, incluyendo:
		\begin{itemize}
			\item Archivo principal de la aplicación (por ejemplo, \texttt{app.py} o \texttt{main.py})
			\item Módulos auxiliares para conexión a base de datos y lógica de negocio
			\item Archivo \texttt{requirements.txt} con todas las dependencias necesarias
			\item Archivo de configuración para la conexión a la base de datos (sin credenciales sensibles)
			\item Instrucciones específicas de ejecución de la aplicación
		\end{itemize}
		
		\item \textbf{README}: Archivo con instrucciones claras que incluyan:
		\begin{itemize}
			\item Estructura del repositorio
			\item Requisitos previos y dependencias
			\item Pasos para configurar la base de datos
			\item Instrucciones para ejecutar la aplicación Streamlit
			\item Credenciales de prueba (si aplica)
			\item Descripción de las funcionalidades implementadas
		\end{itemize}
		
		\item \textbf{Documentos adicionales}: Cualquier otro material que el estudiante considere relevante para la evaluación del proyecto (diagramas, capturas de pantalla de la aplicación, análisis adicionales, pruebas, etc.).
	\end{itemize}
	
	\vspace{2mm}
	
	\textbf{Nota importante}: Asegúrese de que el repositorio incluya un archivo \texttt{.gitignore} apropiado que excluya credenciales, archivos temporales y dependencias que puedan ser reinstaladas.

\end{document}

